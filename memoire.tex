\documentclass[12pt]{article}

\usepackage{geometry}
\usepackage[english]{babel}
\usepackage{blindtext}
\usepackage[]{hyperref}
\hypersetup{colorlinks=false,linkcolor=false,pdfborder = {0 0 0}
}
\geometry{top=2cm,bottom=2cm}
\usepackage{setspace}
\usepackage{graphicx}
\graphicspath{{images/}}
\onehalfspacing
\usepackage{pdfpages}
\usepackage[acronym]{glossaries}
\newacronym{ny}{NY}{New York}
\begin{document}

\newpage
\pagebreak
\hspace{0pt}
\vfill
\begin{center}
\section{Introduction}
\end{center}
\vfill
\hspace{0pt}
\pagebreak

\subsection{Introduction}
Swarm robotics is relatively a new research topic that has gained more attraction in the last few years. It is about  studying how a large number of simple robots (a swarm) can collaborate and work together to achieve predefined objectives and tasks that are often difficult or impossible to do for a single robot.

 
\cite{stats}



\bibliography{bib}
\bibliographystyle{ieeetr}



\end{document}